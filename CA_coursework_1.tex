%%%%%%%%%%%%%%%%%%%%%%%%%%%%%%%%%%%%%%%%%%%%%%%%%%%%%%%%%%%%%%%%%%
%%%%%%%%%%%%%%%%%%%%%%%%%%%%%%%%%%%%%%%%%%%%%%%%%%%%%%%%%%%%%%%%%%
%Packages
\documentclass[10pt, a4paper]{article}
\usepackage[top=3cm, bottom=4cm, left=3.5cm, right=3.5cm]{geometry}
\usepackage{amsmath,amsthm,amsfonts,amssymb,amscd, fancyhdr, color, comment, graphicx, environ, pifont}
\usepackage{float}
\usepackage{mathrsfs}
\usepackage[math-style=ISO]{unicode-math}
\setmathfont{TeX Gyre Termes Math}
\usepackage[dvipsnames]{xcolor}
\usepackage[framemethod=TikZ]{mdframed}
\usepackage{enumerate}
\usepackage[shortlabels]{enumitem}
\usepackage{fancyhdr}
\usepackage{indentfirst}
\usepackage{listings}
\usepackage{sectsty}
\usepackage{thmtools}
\usepackage{shadethm}
\usepackage{hyperref}
\usepackage{setspace}
\usepackage[linguistics]{forest}
\hypersetup{
    colorlinks=true,
    linkcolor=blue,
    filecolor=magenta,      
    urlcolor=blue,
}
%%%%%%%%%%%%%%%%%%%%%%%%%%%%%%%%%%%%%%%%%%%%%%%%%%%%%%%%%%%%%%%%%%
%%%%%%%%%%%%%%%%%%%%%%%%%%%%%%%%%%%%%%%%%%%%%%%%%%%%%%%%%%%%%%%%%%
%Environment setup
\mdfsetup{skipabove=\topskip,skipbelow=\topskip}
\newrobustcmd\ExampleText{%
An \textit{inhomogeneous linear} differential equation has the form
\begin{align}
L[v ] = f,
\end{align}
where $L$ is a linear differential operator, $v$ is the dependent
variable, and $f$ is a given non−zero function of the independent
variables alone.
}
\mdfdefinestyle{theoremstyle}{%
linecolor=black,linewidth=1pt,%
frametitlerule=true,%
frametitlebackgroundcolor=gray!20,
innertopmargin=\topskip,
}
\mdtheorem[style=theoremstyle]{Problem}{Problem}
\newenvironment{Solution}{\textbf{Solution.}}

\definecolor{codegreen}{rgb}{0,0.6,0}
\definecolor{codegray}{rgb}{0.5,0.5,0.5}
\definecolor{codepurple}{rgb}{0.58,0,0.82}
\definecolor{backcolour}{rgb}{0.95,0.95,0.92}

\lstdefinestyle{mystyle}{
    backgroundcolor=\color{backcolour},   
    commentstyle=\color{codegreen},
    keywordstyle=\color{magenta},
    numberstyle=\tiny\color{codegray},
    stringstyle=\color{codepurple},
    basicstyle=\ttfamily\footnotesize,
    breakatwhitespace=false,         
    breaklines=true,                 
    captionpos=b,                    
    keepspaces=true,                 
    numbers=left,                    
    numbersep=5pt,                  
    showspaces=false,                
    showstringspaces=false,
    showtabs=false,                  
    tabsize=2
}

\lstset{style=mystyle}
%%%%%%%%%%%%%%%%%%%%%%%%%%%%%%%%%%%%%%%%%%%%%%%%%%%%%%%%%%%%%%%%%%
%%%%%%%%%%%%%%%%%%%%%%%%%%%%%%%%%%%%%%%%%%%%%%%%%%%%%%%%%%%%%%%%%%
%Fill in the appropriate information below
\newcommand{\norm}[1]{\left\lVert#1\right\rVert}     
\newcommand\course{Course name}                            % <-- course name   
\newcommand\hwnumber{COMP400XXX-X}                                 % <-- homework number
\newcommand\Information{(Your name)}                        % <-- personal information
%%%%%%%%%%%%%%%%%%%%%%%%%%%%%%%%%%%%%%%%%%%%%%%%%%%%%%%%%%%%%%%%%%
%%%%%%%%%%%%%%%%%%%%%%%%%%%%%%%%%%%%%%%%%%%%%%%%%%%%%%%%%%%%%%%%%%
%Page setup
\pagestyle{fancy}
\headheight 35pt
\lhead{\today}
\rhead{\includegraphics[width=2.5cm]{icl_logo.png}}
\lfoot{}
\pagenumbering{arabic}
\cfoot{\small\thepage}
\rfoot{}
\headsep 1.2em
\renewcommand{\baselinestretch}{1.25}
%%%%%%%%%%%%%%%%%%%%%%%%%%%%%%%%%%%%%%%%%%%%%%%%%%%%%%%%%%%%%%%%%%
%%%%%%%%%%%%%%%%%%%%%%%%%%%%%%%%%%%%%%%%%%%%%%%%%%%%%%%%%%%%%%%%%%
%Add new commands here
\renewcommand{\labelenumi}{\alph{enumi})}
\newcommand{\Z}{\mathbb Z}
\newcommand{\R}{\mathbb R}
\newcommand{\Q}{\mathbb Q}
\newcommand{\NN}{\mathbb N}
\newcommand{\PP}{\mathbb P}
\DeclareMathOperator{\Mod}{Mod} 
\renewcommand\lstlistingname{Algorithm}
\renewcommand\lstlistlistingname{Algorithms}
\def\lstlistingautorefname{Alg.}
\newtheorem*{theorem}{Theorem}
\newtheorem*{lemma}{Lemma}
\newtheorem{case}{Case}
\newcommand{\assign}{:=}
\newcommand{\infixiff}{\text{ iff }}
\newcommand{\nobracket}{}
\newcommand{\backassign}{=:}
\newcommand{\tmmathbf}[1]{\ensuremath{\boldsymbol{#1}}}
\newcommand{\tmop}[1]{\ensuremath{\operatorname{#1}}}
\newcommand{\tmtextbf}[1]{\text{{\bfseries{#1}}}}
\newcommand{\tmtextit}[1]{\text{{\itshape{#1}}}}

\newenvironment{itemizedot}{\begin{itemize} \renewcommand{\labelitemi}{$\bullet$}\renewcommand{\labelitemii}{$\bullet$}\renewcommand{\labelitemiii}{$\bullet$}\renewcommand{\labelitemiv}{$\bullet$}}{\end{itemize}}
\catcode`\<=\active \def<{
\fontencoding{T1}\selectfont\symbol{60}\fontencoding{\encodingdefault}}
\catcode`\>=\active \def>{
\fontencoding{T1}\selectfont\symbol{62}\fontencoding{\encodingdefault}}
\catcode`\<=\active \def<{
\fontencoding{T1}\selectfont\symbol{60}\fontencoding{\encodingdefault}}

%%%%%%%%%%%%%%%%%%%%%%%%%%%%%%%%%%%%%%%%%%%%%%%%%%%%%%%%%%%%%%%%%%
%%%%%%%%%%%%%%%%%%%%%%%%%%%%%%%%%%%%%%%%%%%%%%%%%%%%%%%%%%%%%%%%%%
%Begin now!



\begin{document}

\begin{titlepage}
    \begin{center}
        \vspace*{3cm}
            
        \Huge
        \textbf{
            Autumn Take-home Assessment
                }
            
            
        \vspace{1.5cm}
        \Large
            
        \textbf{
        CID number: Once upon a time...}% <-- author
        
            
        \vfill
        
        MATH40004: Calculus and Applications        
        \vspace{1cm}
            
        \includegraphics[width=0.4\textwidth]{icl_logo.png}
        \\
        
        \Large
        
        \today
            
    \end{center}
\end{titlepage}


\newpage
\begin{Problem}
    The equation

    $$
    y^{\prime}=1+y^{2}, \quad y(0)=0,
    $$
    
    can be solved using separation of variables and integration to find $y=\tan x$.
    
    (a) Find a power series solution of $(* *)$ and hence show that
    
    $$
    \tan x=x+\frac{1}{3} x^{3}+\frac{2}{15} x^{5}+\ldots
    $$
    
    [Note: you will need to use the earlier formulas for multiplication of two infinite power series.]
    
    (b) Now get the result above by repeated differentiation of $(* *)$ and use of the formula $a_{n}=\frac{f^{(n)}(0)}{n !}$
\end{Problem}
\begin{Solution}
    (a) We assume that a power series solution exists, i.e.

$$
y(x)=\sum_{n=0}^{\infty} a_{n} x^{n}=a_{0}+a_{1} x+a_{2} x^{2}+\ldots+a_{n} x^{n}+\ldots
$$

converges for $|x|<R$ for some positive radius of convergence $R>0$. Then, we can differentiate the power series term by term to find

$$
y(x)^{\prime}=\sum_{n=1}^{\infty} n a_{n} x^{n-1}=a_{1}+2 a_{2} x+3 a_{3} x^{2}+\ldots+(n+1) a_{n+1} x^{n}+\ldots
$$

Hence $(* *)$ is satisfied if $(2)=1+(1)^{2}$. Therefore, we have

$$
\sum_{n=1}^{\infty} n a_{n} x^{n-1}=1+\left(\sum_{n=0}^{\infty} a_{n} x^{n}\right)^{2}
$$

by the formulas for multiplication of two infinite power series, we get:

$$
\left(\sum_{n=0}^{\infty} a_{n} x^{n}\right)^{2}=\sum_{n=0}^{\infty}\left(\sum_{m=0}^{n} a_{m} a_{n-m}\right) x^{n}
$$

Substitute (4) into (3), we get:

$$
\sum_{n=1}^{\infty} n a_{n} x^{n-1}=1+\sum_{n=0}^{\infty}\left(\sum_{m=0}^{n} a_{m} a_{n-m}\right) x^{n} .
$$

Note that $a_{0}=0$ by the initial condition $y(0)=0$. If the equation (5) is satisfied, the coefficients of different powers of $x$ must match. By inspection

$$
\begin{aligned}
a_{0} & =0 \\
a_{1} & =1+a_{0}^{2} \Rightarrow a_{1}=1 \\
2 a_{2} & =2 a_{0} a_{1} \Rightarrow a_{2}=0 \\
3 a_{3} & =2 a_{0} a_{2}+a_{1}^{2} \Rightarrow a_{3}=\frac{1}{3} \\
4 a_{4} & =2 a_{0} a_{3}+2 a_{1} a_{2} \Rightarrow a_{4}=0 \\
5 a_{5} & =2 a_{0} a_{4}+2 a_{1} a_{3}+a_{2}^{2} \Rightarrow a_{5}=\frac{2}{15} \\
& \vdots \\
n a_{n} & =\sum_{m=0}^{n-1} a_{m} a_{(n-1)-m} \\
(n+1) a_{n+1} & =\sum_{m=0}^{n} a_{m} a_{n-m}
\end{aligned}
$$

Therefore, we get a power series solution of $(* *)$, which satisfied the recursion formula (6).

(**) can be solved using separation of variable and integration to find $y=\tan x$. The detail is followed:

$$
\begin{aligned}
y^{\prime}=1 & +y^{2} \\
\frac{d y}{d x} & =1+y^{2} \\
\frac{1}{1+y^{2}} d y & =d x \\
\int \frac{1}{1+y^{2}} d y & =\int d x \\
\tan ^{-1} y & =x+C_{1} \\
y & =\tan x+C
\end{aligned}
$$

$y(0)=0$, then $C=0$. Therefore, $y=\tan x$ is also a solution of $(* *)$. Then we have

$$
y(x)=\tan x=\sum_{n=0}^{\infty} a_{0} x^{n}=x+\frac{1}{3} x^{3}+\frac{2}{15} x^{5}+\ldots
$$

(b) We have $y^{\prime}=1+y^{2} \Longrightarrow f^{\prime}(0)=1$ as $f(0)=1$. Then, differentiate $(* *)$ at the both sides and repeat:

$$
\begin{aligned}
y^{\prime \prime} & =2 y y^{\prime} \Longrightarrow f^{\prime \prime}(0)=0 \\
y^{(3)} & =2\left(y^{\prime}\right)^{2}+2 y y^{\prime \prime} \Longrightarrow f^{(3)}(0)=2 \\
y^{(4)} & =4 y^{\prime} y^{\prime \prime}+2 y^{\prime} y^{\prime \prime}+2 y y^{(3)} \Longrightarrow f^{(4)}(0)=0 \\
y^{(5)} & =4\left(y^{\prime \prime}\right)^{2}+4 y^{\prime} y^{(3)}+2\left(y^{\prime \prime}\right)^{2}+2 y^{\prime} y^{(3)}+2 y^{\prime} y(3)+2 y y^{(4)} \Rightarrow f^{(5)}(0)=16
\end{aligned}
$$$$
\vdots
$$ 

Use the formula $a_{n}=\frac{f^{(n)}(0)}{n !}$

$$
\begin{aligned}
& a_{0}=\frac{f(0)}{0 !}=0 \\
& a_{1}=\frac{f^{\prime}(0)}{1 !}=1 \\
& a_{2}=\frac{f^{\prime \prime}(0)}{2 !}=0 \\
& a_{3}=\frac{f^{(3)(0)}}{3 !}=\frac{1}{3} \\
& a_{4}=\frac{f^{(4)}(0)}{4 !}=0 \\
& a_{5}=\frac{f^{(5)}(0)}{5 !}=\frac{2}{15}
\end{aligned}
$$

Then we get

$$
\tan x=x+\frac{1}{3} x^{3}+\frac{2}{15} x^{5}+\ldots
$$
\end{Solution}
%%%%%%%%%%%%%%%%%%%%%%%%%%%%%%%%%%%%%%%%%%%%%%%%%%%%%%%%%%%%%%%%%%
%Complete the assignment now
\end{document}
%%%%%%%%%%%%%%%%%%%%%%%%%%%%%%%%%%%%%%%%%%%%%%%%%%%%%%%%%%%%%%%%%%
%%%%%%%%%%%%%%%%%%%%%%%%%%%%%%%%%%%%%%%%%%%%%%%%%%%%%%%%%%%%%%%%%%